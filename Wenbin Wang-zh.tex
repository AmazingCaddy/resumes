%!TEX TS-program = xelatex  
%!TEX encoding = UTF-8 Unicode

\documentclass[a4paper]{article}
\usepackage{fullpage}
\usepackage[top=0.7in,bottom=1.0in,left=1.2in,right=1.2in]{geometry}
\usepackage{amsmath}
\usepackage{amssymb}
\usepackage[usenames]{color}

% begin font setting
\usepackage{fontspec}								% fonespec 用于支持中文,选择一个中文字体
\usepackage{xunicode}								% 支持Unicode
\usepackage{xltxtra}								% for XeLaTex User \XeLaTex

\defaultfontfeatures{Mapping=tex-text}
\setromanfont{宋体}								% 设置中文字体
%\setmainfont{Kai}								% 默认字体 
\setmonofont{Consolas}								% 默认等宽字体
\setsansfont{Arial}									% 英文无衬线字体
\XeTeXlinebreaklocale “zh”
\XeTeXlinebreakskip = 0.1pt plus 1pt minus 1.5pt		%文章内中文自动换行

%设置新字体
\newfontfamily{\fontcourier}{Courier New} 			% courier New
%\newfontfamily{\fontkai}{KaiTi}					% 楷体,我觉得最漂亮的中文字体
%\newfontfamily{\fontsong}{SimSun}					% 宋体,很难看但是很常用的中文字体
%\newfontfamily{\fonthei}{SimHei}					% 黑体
\newfontfamily{\fontarial}{Arial}					% Arial
%\newfontfamily{\fonthelvetica}{Helvetica}			% Helvetica
\newfontfamily{\fontconsolas}{Consolas}				% Consolas
%\newfontfamily{\fonttimes}{Times}					% Times
\newfontfamily{\fontmenlo}{Menlo}					% Menlo
%\newfontfamily{\fontgillsnas}{Gill Sans}				% Gill Sans

% end of font setting

\raggedright

\pagenumbering{arabic}

\def\bull{\vrule height 0.8ex width .7ex depth -.1ex }
% DEFINITIONS FOR RESUME

\newenvironment{changemargin}[2]{%
  \begin{list}{}{%
    \setlength{\topsep}{0pt}%
    \setlength{\leftmargin}{#1}%
    \setlength{\rightmargin}{#2}%
    \setlength{\listparindent}{\parindent}%
    \setlength{\itemindent}{\parindent}%
    \setlength{\parsep}{\parskip}%
  }%
  \item[]}{\end{list}
}

\newcommand{\lineover}{
	\begin{changemargin}{-0.05in}{-0.05in}
		\vspace*{-8pt}
		\hrulefill \\
		\vspace*{-2pt}
	\end{changemargin}
}

\newcommand{\header}[1]{
	\begin{changemargin}{-0.5in}{-0.5in}
		\scshape{\textbf{#1}}\\
  	%\lineover
	\end{changemargin}
}

\newcommand{\contact}[3]{
	\begin{changemargin}{-0.5in}{-0.5in}
		{\Large \scshape {#1}}\\ \smallskip
		\lineover
		\begin{flushright}
			\emph{\fontarial #2}\\ \smallskip
			\emph{\fontarial #3}\smallskip
		\end{flushright}
	\end{changemargin}
}

\newenvironment{body} {
	\vspace*{-16pt}
	\begin{changemargin}{-0.25in}{-0.5in}
  }	
	{\end{changemargin}
}	

\newcommand{\school}[4]{
	\textbf{#1} \hfill \emph{#2\\}
	#3\\ 
	#4\\
}

% END RESUME DEFINITIONS

\begin{document}

%%%%%%%%%%%%%%%%%%%%%%%%%%%%%%%%%%%%%%%%%%%%%%%%%%%%%%%%%%%%%%%%%%%%%%%%%%%%%%%%
% Name
\contact{王文斌}{ecnuwbwang@gmail.com}{(+86)151-2101-7457}

\renewcommand{\baselinestretch}{1.2} \normalsize

%%%%%%%%%%%%%%%%%%%%%%%%%%%%%%%%%%%%%%%%%%%%%%%%%%%%%%%%%%%%%%%%%%%%%%%%%%%%%%%%
% 教育背景
\header{教育背景}

\begin{body}
	\vspace{14pt}
	{华东师范大学 计算机应用技术 硕士研究生} \hfill \emph{\fontarial 2012 - }至今 \\
	%\smallskip
	{华东师范大学 计算机科学与技术 本科} \hfill \emph{\fontarial 2008 - 2012} \\
\end{body}

\medskip


%%%%%%%%%%%%%%%%%%%%%%%%%%%%%%%%%%%%%%%%%%%%%%%%%%%%%%%%%%%%%%%%%%%%%%%%%%%%%%%%
% 获奖经历
\header{获奖经历}
\begin{body}
	\vspace{14pt}
	{\emph{\fontarial ACM/ICPC} 国际大学生程序设计竞赛福州赛区 铜奖} \hfill \emph{\fontarial 2011/11}\\
	%\smallskip
	{\emph{\fontarial ACM/ICPC} 国际大学生程序设计竞赛北京赛区 铜奖} \hfill \emph{\fontarial 2011/10}\\
	%\smallskip
	{\emph{\fontarial MCM} 美国大学生数学建模大赛 二等奖} \hfill \emph{\fontarial 2011/02}\\
	%\smallskip
	{\emph{\fontarial ACM/ICPC} 国际大学生程序设计竞赛天津赛区 铜奖} \hfill \emph{\fontarial 2010/10}\\
\end{body}

\medskip

% \newpage{} % uncomment this line if you want to force a new page


%%%%%%%%%%%%%%%%%%%%%%%%%%%%%%%%%%%%%%%%%%%%%%%%%%%%%%%%%%%%%%%%%%%%%%%%%%%%%%%%
% 工作经历
\header{工作经历}

\begin{body}
	\vspace{14pt}
	\textbf{百度} \hfill 实习生\\ 
	%\smallskip
	地点:上海 \hfill \emph{\fontarial 2013/07 - 2013/10}\\ 
	%\smallskip
	参与百度广告管家2.0版本业务端的部分测试工作,从测试的角度为整个项目提供支持。
	%\smallskip
	
	\vspace{6pt}
	\textbf{谷歌-企业社会责任部} \hfill 实习生\\ 
	%\smallskip
	地点:上海 \hfill \emph{\fontarial 2012/02 - 2012/08}\\ 
	%\smallskip
	主要负责 \emph{\fontarial Java+MySQL} 以及 \emph{\fontarial PHP+MySQL} 的网站开发工作。\\ 
	%\smallskip
	\vspace*{-6pt}
	\begin{itemize} \itemsep -0pt  % reduce space between items
		\item 负责网站日常维护和功能升级。\\
	\end{itemize}
	\vspace*{-12pt}
	\begin{itemize} \itemsep -0pt  % reduce space between items
		\item 参与完成网站的迁移,从\emph{\fontarial Java Web}迁移到\emph{\fontarial PHP}。\\
	\end{itemize}
	\vspace*{-12pt}
	\begin{itemize} \itemsep -0pt  % reduce space between items
		\item 参与项目\\
		% \smallskip
		\textbf{益暖中华} \emph{\fontarial(www.gong1chuang1.com)} 一个提供平台,征集创意,资助获奖团队进行创意实施的大学生公益大赛。\\
	\end{itemize}
\end{body}

\medskip

%%%%%%%%%%%%%%%%%%%%%%%%%%%%%%%%%%%%%%%%%%%%%%%%%%%%%%%%%%%%%%%%%%%%%%%%%%%%%%%%
% 项目经历
\header{项目经历}

\begin{body}
	\vspace{14pt}
	
	{\textbf{自动查询扩展技术研究 研究课题}}{} \hfill \emph{\fontarial 2014/03 - }至今\\
	%\smallskip
	项目描述:本课题目标是通过查询扩展技术,提高信息检索的准确率。\\
	%\medskip
	\vspace*{-6pt}
	\begin{itemize} \itemsep -0pt  % reduce space between items
		\item 参与研究查询词权重调整\emph{\fontarial(query reweighting)}技术,提出了一种基于全局分析的权重计算公式。\\
	\end{itemize}
	\vspace*{-12pt}
	\begin{itemize} \itemsep -0pt  % reduce space between items
		\item 技术环境:\emph{\fontarial Mac Indri}\\
	\end{itemize}
	
	{\textbf{基于本体的语意搜索引擎的开发与研究 研究课题}}{} \hfill \emph{\fontarial 2012/12 - 2013/06}\\
	%\smallskip
	项目描述:本课题目标是将语义分析的技术引入到搜索引擎,改善当前搜索引擎的搜索效果,提出基于本体的语意搜索引擎模型,来提高搜索结果的准确率。\\
	%\medskip
	\vspace*{-6pt}
	\begin{itemize} \itemsep -0pt  % reduce space between items
		\item 参与搜索引擎模块的研究,除了使用现成的\emph{\fontarial Google}和\emph{\fontarial Bing},采用\emph{\fontarial Hadoop}云环境和\emph{\fontarial Nutch}搜索引擎。\\
	\end{itemize}
	\vspace*{-12pt}
	\begin{itemize} \itemsep -0pt  % reduce space between items
		\item 技术环境:\emph{\fontarial Linux Nutch Hadoop}\\
	\end{itemize}
	
	{\textbf{Small C编译器 课程设计}}{} \hfill \emph{\fontarial 2011/10 - 2011/12}\\
	项目描述:根据文法,使用\emph{\fontarial LL(1)}方法,实现一个\emph{\fontarial Small C}编译器。\\
	\vspace*{-6pt}
	\begin{itemize} \itemsep -0pt  % reduce space between items
		\item 在课程要求的基础上,新增了一些语法的实现,有比较完善的编译报错机制。\\
	\end{itemize}
	\vspace*{-12pt}
	\begin{itemize} \itemsep -0pt  % reduce space between items
		\item 技术环境:\emph{\fontarial Java}\\
	\end{itemize}
	
%	{\textbf{数独游戏 课程设计}}{} \hfill \emph{\fontmenlo2010/10 - 2010/12}\\
%	项目描述:在传统的 {\fontmenlo 9×9} 数独游戏基础上,定制规则。该小软件可以自动生成游戏局面,提供答案,还支持手动输入数独,帮助用户解答。\\
%	\vspace*{-6pt}
%	\begin{itemize} \itemsep -0pt  % reduce space between items
%		\item 负责游戏界面的设计开发,后台解答模块算法实现。\\
%	\end{itemize}
%	\vspace*{-12pt}
%	\begin{itemize} \itemsep -0pt  % reduce space between items
%		\item 技术环境:\emph{\fontmenlo C\raise.3ex\hbox{\small{\#}} .NET}\\
%	\end{itemize}
\end{body}

\medskip

%%%%%%%%%%%%%%%%%%%%%%%%%%%%%%%%%%%%%%%%%%%%%%%%%%%%%%%%%%%%%%%%%%%%%%%%%%%%%%%%
% 个人技能
\header{个人技能}

\begin{body}
	\vspace{14pt}
	具有 {\fontarial Web} 开发基础和经验\\
	较好的掌握了算法和数据结构\\
	编程语言:\emph{\fontarial C/C\raise.3ex\hbox{\small++}}熟练 \emph{\fontarial PHP}熟练 \emph{\fontarial Java}一般 \emph{\fontarial Python}一般\\
	英语技能:\emph{\fontarial CET-6}\\
\end{body}

\medskip


%%%%%%%%%%%%%%%%%%%%%%%%%%%%%%%%%%%%%%%%%%%%%%%%%%%%%%%%%%%%%%%%%%%%%%%%%%%%%%%%
% 兴趣爱好
\header{兴趣爱好}

\begin{body}
	\vspace{14pt}
	乒乓球 电影\\
\end{body}

\end{document}
