%!TEX TS-program = xelatex  
%!TEX encoding = UTF-8 Unicode

\documentclass[a4paper]{article}
\usepackage{fullpage}
\usepackage[top=0.7in,bottom=1.0in,left=1.3in,right=1.3in]{geometry}
\usepackage{amsmath}
\usepackage{amssymb}
\usepackage[usenames]{color}

% begin font setting
\usepackage{fontspec}								% fonespec 用于支持中文,选择一个中文字体
\usepackage{xunicode}								% 支持Unicode
\usepackage{xltxtra}								% for XeLaTex User \XeLaTex

\defaultfontfeatures{Mapping=tex-text}
\setromanfont{微软雅黑}								% 设置中文字体
%\setmainfont{Kai}								% 默认字体 
\setmonofont{Consolas}								% 默认等宽字体
\setsansfont{Arial}									% 英文无衬线字体
\XeTeXlinebreaklocale “zh”
\XeTeXlinebreakskip = 0.1pt plus 1pt minus 1.5pt		%文章内中文自动换行

%设置新字体
\newfontfamily{\fontcourier}{Courier New} 			% courier New
%\newfontfamily{\fontkai}{KaiTi}					% 楷体,我觉得最漂亮的中文字体
%\newfontfamily{\fontsong}{SimSun}					% 宋体,很难看但是很常用的中文字体
%\newfontfamily{\fonthei}{SimHei}					% 黑体
\newfontfamily{\fontarial}{Arial}					% Arial
%\newfontfamily{\fonthelvetica}{Helvetica}			% Helvetica
\newfontfamily{\fontconsolas}{Consolas}				% Consolas
%\newfontfamily{\fonttimes}{Times}					% Times
\newfontfamily{\fontmenlo}{Menlo}					% Menlo
%\newfontfamily{\fontgillsnas}{Gill Sans}				% Gill Sans

% end of font setting

\raggedright

\pagenumbering{arabic}

\def\bull{\vrule height 0.8ex width .7ex depth -.1ex }
% DEFINITIONS FOR RESUME

\newenvironment{changemargin}[2]{%
  \begin{list}{}{%
    \setlength{\topsep}{0pt}%
    \setlength{\leftmargin}{#1}%
    \setlength{\rightmargin}{#2}%
    \setlength{\listparindent}{\parindent}%
    \setlength{\itemindent}{\parindent}%
    \setlength{\parsep}{\parskip}%
  }%
  \item[]}{\end{list}
}

\newcommand{\lineover}{
	\begin{changemargin}{-0.05in}{-0.05in}
		\vspace*{-8pt}
		\hrulefill \\
		\vspace*{-2pt}
	\end{changemargin}
}

\newcommand{\header}[1]{
	\begin{changemargin}{-0.5in}{-0.5in}
	\fontsize{12}{14} \scshape{\textbf{#1}}\\
  	%\lineover
	\end{changemargin}
}

\newcommand{\contact}[3]{
	\begin{changemargin}{-0.5in}{-0.5in}
		{\fontsize{15}{18} \scshape \textbf{#1}}\\ \smallskip
		\lineover
		\begin{flushright}
			{\fontarial \emph{#2}}\\ \smallskip
			{\fontarial \emph{#3}}\smallskip
		\end{flushright}
	\end{changemargin}
}

\newenvironment{body} {
	\vspace*{-16pt}
	\begin{changemargin}{-0.5in}{-0.5in}
  }	
	{\end{changemargin}
}	

\newcommand{\school}[4]{
	\textbf{#1} \hfill \emph{#2\\}
	#3\\ 
	#4\\
}

% END RESUME DEFINITIONS

\begin{document}

%%%%%%%%%%%%%%%%%%%%%%%%%%%%%%%%%%%%%%%%%%%%%%%%%%%%%%%%%%%%%%%%%%%%%%%%%%%%%%%%
% Name
\contact{王文斌}{ecnuwbwang@gmail.com}{(+86)151-2101-7457}

\renewcommand{\baselinestretch}{1.2} \normalsize

%%%%%%%%%%%%%%%%%%%%%%%%%%%%%%%%%%%%%%%%%%%%%%%%%%%%%%%%%%%%%%%%%%%%%%%%%%%%%%%%
% 教育经历
\header{教育经历}

\begin{body}
	\vspace{14pt}
	{华东师范大学 计算机应用技术 硕士研究生} \hfill {\fontarial 2012/09 - 2015/06} \\
	%\smallskip
	{华东师范大学 计算机科学与技术 本科} \hfill {\fontarial 2008/09 - 2012/06} \\
\end{body}

\medskip

%%%%%%%%%%%%%%%%%%%%%%%%%%%%%%%%%%%%%%%%%%%%%%%%%%%%%%%%%%%%%%%%%%%%%%%%%%%%%%%%
% 项目经历
\header{项目经历}

\begin{body}
	\vspace{16pt}
	\textbf{微软} \hfill 软件开发工程师\\ 
	%\smallskip
	\textbf{\fontarial UCM(Unified Customer Management)} \hfill {\fontarial 2015/07 - NOW}\\ 
	%\smallskip
	{\fontarial UCM}是{\fontarial Bing Ads}的统一用户管理工具,主要供广告销售团队使用,以此来服务广告主。{\fontarial UCM}是一个{\fontarial Web}产品,整个产品的架构:\\
	\vspace*{-6pt}
	\begin{itemize} \itemsep -0pt  % reduce space between items
		\item 前端是一个{\fontarial SPA},使用{\fontarial TypeScript}开发,通过{\fontarial Ajax}调用从后端获取数据。\\
	\end{itemize}
	\vspace*{-12pt}
	\begin{itemize} \itemsep -0pt  % reduce space between items
		\item 后端是一个{\fontarial ASP.Net MVC}服务,会从{\fontarial UCM}的数据库和第三方服务获取数据,进行计算,返回给前端。\\
	\end{itemize}
	\vspace*{-12pt}
	\begin{itemize} \itemsep -0pt  % reduce space between items
		\item {\fontarial Azure Data Factory}同步数据到{\fontarial UCM}的数据库。\\
	\end{itemize}
	%\smallskip
	\vspace*{-6pt}
	本人的主要工作内容:\\ 
	%\smallskip
	\vspace*{-6pt}
	\begin{itemize} \itemsep -0pt  % reduce space between items
		\item 前端的功能开发、维护和性能优化,参与优化{\fontarial SPA First Load Time}过长,局部刷新性能优化。\\
	\end{itemize}
	\vspace*{-12pt}
	\begin{itemize} \itemsep -0pt  % reduce space between items
		\item 基于{\fontarial UCM}前端的代码框架,增强了前端单元测试框架,降低开发成本。\\
	\end{itemize}
	\vspace*{-12pt}
	\begin{itemize} \itemsep -0pt  % reduce space between items
		\item 参与前端技术更新,将{\fontarial React}和{\fontarial UCM}的现有代码整合,使它们能够一起工作。\\
	\end{itemize}
	\vspace*{-12pt}
	\begin{itemize} \itemsep -0pt  % reduce space between items
		\item 作为{\fontarial sub lead (2018/05 -)},带领两个同事,负责和{\fontarial PM}沟通需求,执行前端开发。\\
	\end{itemize}
	\vspace*{-2pt}
	\textbf{智能客服多轮对话系统} \hfill {\fontarial 2018/06 - 2018/09}\\ 
	该系统是基于意图识别和填槽技术的面向任务的对话系统,采用微服务架构,包含对话任务设计、对话模板管理和对话逻辑处理三部分。该系统提供{\fontarial Web UI}允许管理员自定义对话任务模板,并提供聊天窗口让用户与机器人进行多轮对话。本人负责开发对话模板管理的{\fontarial Web UI}部分,使用{\fontarial React}开发。\\
	\vspace{10pt}
	\textbf{百度} \hfill 测试实习生\\ 
	%\smallskip
	\textbf{广告管家} \hfill {\fontarial 2013/07 - 2013/10}\\ 
	%\smallskip
	参与百度广告管家{\fontarial 2.0}版本业务端的部分测试工作,使用{\fontarial Selenium}完成自动化测试。
	%\smallskip

	\vspace{10pt}
	\textbf{谷歌-企业社会责任部} \hfill 实习生\\ 
	%\smallskip
	\textbf{益暖中华{\fontarial (G1C1)}} \hfill {\fontarial 2012/02 - 2012/08}\\ 
	%\smallskip
	主要负责{\fontarial Java+MySQL}以及{\fontarial PHP+MySQL}的网站开发工作。\\ 
	%\smallskip
	\vspace*{-6pt}
	\begin{itemize} \itemsep -0pt  % reduce space between items
		\item 负责益暖中华{\fontarial(www.gong1chuang1.com)}网站日常维护和功能升级。\\
	\end{itemize}
	\vspace*{-12pt}
	\begin{itemize} \itemsep -0pt  % reduce space between items
		\item 参与完成益暖中华主网站的迁移,从{\fontarial Java Web}迁移到{\fontarial PHP}。\\
	\end{itemize}
\end{body}

\medskip


%%%%%%%%%%%%%%%%%%%%%%%%%%%%%%%%%%%%%%%%%%%%%%%%%%%%%%%%%%%%%%%%%%%%%%%%%%%%%%%%
% 获奖经历
\header{获奖经历}
\begin{body}
	\vspace{14pt}
	{{\fontarial ACM/ICPC} 国际大学生程序设计竞赛福州赛区 铜奖} \hfill {\fontarial 2011/11}\\
	%\smallskip
	{{\fontarial ACM/ICPC} 国际大学生程序设计竞赛北京赛区 铜奖} \hfill {\fontarial 2011/10}\\
	%\smallskip
	{{\fontarial MCM} 美国大学生数学建模大赛 二等奖} \hfill {\fontarial 2011/02}\\
	%\smallskip
%	{{\fontarial ACM/ICPC} 国际大学生程序设计竞赛天津赛区 铜奖} \hfill {\fontarial 2010/10}\\
\end{body}

\medskip


%%%%%%%%%%%%%%%%%%%%%%%%%%%%%%%%%%%%%%%%%%%%%%%%%%%%%%%%%%%%%%%%%%%%%%%%%%%%%%%%
% 项目经历
%\header{项目经历}

%\begin{body}
%	\vspace{14pt}
	
	%{\textbf{自动查询扩展技术研究 研究课题}}{} \hfill \emph{\fontarial 2014/03 - }至今\\
	%\smallskip
	%项目描述:本课题目标是通过查询扩展技术,提高信息检索的准确率。\\
	%\medskip
	%\vspace*{-6pt}
	%\begin{itemize} \itemsep -0pt  % reduce space between items
	%	\item 参与研究查询词权重调整\emph{\fontarial(query reweighting)}技术,提出了一种基于全局分析的权重计算公式。\\
	%\end{itemize}
	%\vspace*{-12pt}
	%\begin{itemize} \itemsep -0pt  % reduce space between items
	%	\item 技术环境:\emph{\fontarial Mac Indri}\\
	%\end{itemize}
	
%	{\textbf{基于本体的语意搜索引擎的开发与研究 研究课题}}{} \hfill {\fontarial 2012/12 - 2013/06}\\
	%\smallskip
%	项目描述:本课题目标是将语义分析的技术引入到搜索引擎,改善当前搜索引擎的搜索效果,提出基于本体的语意搜索引擎模型,来提高搜索结果的准确率。\\
	%\medskip
%	\vspace*{-6pt}
%	\begin{itemize} \itemsep -0pt  % reduce space between items
%		\item 参与搜索引擎模块的研究,除了使用现成的{\fontarial Google}和{\fontarial Bing},采用{\fontarial Hadoop}云环境和{\fontarial Nutch}搜索引擎。\\
%	\end{itemize}
%	\vspace*{-12pt}
%	\begin{itemize} \itemsep -0pt  % reduce space between items
%		\item 技术环境:{\fontarial Linux Nutch Hadoop}\\
%	\end{itemize}
	
%	{\textbf{{\fontarial Small C}编译器 课程设计}}{} \hfill {\fontarial 2011/10 - 2011/12}\\
%	项目描述:根据文法,使用{\fontarial LL(1)}方法,实现一个{\fontarial Small C}编译器。\\
%	\vspace*{-6pt}
%	\begin{itemize} \itemsep -0pt  % reduce space between items
%		\item 在课程要求的基础上,新增了一些语法的实现,有比较完善的编译报错机制。\\
%	\end{itemize}
%	\vspace*{-12pt}
%	\begin{itemize} \itemsep -0pt  % reduce space between items
%		\item 技术环境:{\fontarial Java}\\
%	\end{itemize}
	
%	{\textbf{数独游戏 课程设计}}{} \hfill \emph{\fontmenlo2010/10 - 2010/12}\\
%	项目描述:在传统的 {\fontmenlo 9×9} 数独游戏基础上,定制规则。该小软件可以自动生成游戏局面,提供答案,还支持手动输入数独,帮助用户解答。\\
%	\vspace*{-6pt}
%	\begin{itemize} \itemsep -0pt  % reduce space between items
%		\item 负责游戏界面的设计开发,后台解答模块算法实现。\\
%	\end{itemize}
%	\vspace*{-12pt}
%	\begin{itemize} \itemsep -0pt  % reduce space between items
%		\item 技术环境:\emph{\fontmenlo C\raise.3ex\hbox{\small{\#}} .NET}\\
%	\end{itemize}
%\end{body}

%\medskip

% \newpage{} % uncomment this line if you want to force a new page

%%%%%%%%%%%%%%%%%%%%%%%%%%%%%%%%%%%%%%%%%%%%%%%%%%%%%%%%%%%%%%%%%%%%%%%%%%%%%%%%
% 自我评价
\header{自我评价}

\begin{body}
	\vspace{14pt}
	具有{\fontarial 2-3}年{\fontarial Web}前端开发经验,涉及过后端的开发,掌握了常用的算法和数据结构。\\
	做事专注,喜欢钻研,相信技术改变世界。\\
	编程语言:{\fontarial TypeScript/JavaScript/ES6}(熟练) {\fontarial C/C\raise.3ex\hbox{\small++}/Python/Java/PHP}(熟悉)\\
	英语技能:{\fontarial CET-6}\\
\end{body}

\medskip

%%%%%%%%%%%%%%%%%%%%%%%%%%%%%%%%%%%%%%%%%%%%%%%%%%%%%%%%%%%%%%%%%%%%%%%%%%%%%%%%
% 自我评价
%\header{自我评价}

%\begin{body}
%	\vspace{14pt}
%	做事专注,喜欢钻研,相信技术改变世界。\\
%	乐观开朗,热爱生活,一个发散正能量的人。\\
%\end{body}

%\medskip


%%%%%%%%%%%%%%%%%%%%%%%%%%%%%%%%%%%%%%%%%%%%%%%%%%%%%%%%%%%%%%%%%%%%%%%%%%%%%%%%
% 兴趣爱好
\header{兴趣爱好}

\begin{body}
	\vspace{14pt}
	烹饪 羽毛球 电影\\
\end{body}

\end{document}
