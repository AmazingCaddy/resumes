%!TEX TS-program = xelatex  
%!TEX encoding = UTF-8 Unicode

\documentclass[a4paper]{article}
\usepackage{fullpage}
\usepackage[top=0.7in,bottom=1.0in,left=1.3in,right=1.3in]{geometry}
\usepackage{amsmath}
\usepackage{amssymb}
\usepackage[usenames]{color}
\usepackage{hyperref}
\usepackage{ragged2e}
\makeatletter
\def\UrlAlphabet{%
      \do\a\do\b\do\c\do\d\do\e\do\f\do\g\do\h\do\i\do\j%
      \do\k\do\l\do\m\do\n\do\o\do\p\do\q\do\r\do\s\do\t%
      \do\u\do\v\do\w\do\x\do\y\do\z\do\A\do\B\do\C\do\D%
      \do\E\do\F\do\G\do\H\do\I\do\J\do\K\do\L\do\M\do\N%
      \do\O\do\P\do\Q\do\R\do\S\do\T\do\U\do\V\do\W\do\X%
      \do\Y\do\Z}
\def\UrlDigits{\do\1\do\2\do\3\do\4\do\5\do\6\do\7\do\8\do\9\do\0}
\g@addto@macro{\UrlBreaks}{\UrlOrds}
\g@addto@macro{\UrlBreaks}{\UrlAlphabet}
\g@addto@macro{\UrlBreaks}{\UrlDigits}
\makeatother

% begin font setting
\usepackage{fontspec}								% fonespec 用于支持中文,选择一个中文字体
\usepackage{xunicode}								% 支持Unicode
\usepackage{xltxtra}								% for XeLaTex User \XeLaTex

\defaultfontfeatures{Mapping=tex-text}
\setromanfont{PingFang SC}								% 设置中文字体
\setmainfont{PingFang SC}								% 默认字体 
%\setmonofont{Consolas}								% 默认等宽字体
\setsansfont{Arial}									% 英文无衬线字体
\XeTeXlinebreaklocale “zh”
\XeTeXlinebreakskip = 0.1pt plus 1pt minus 1.5pt		%文章内中文自动换行

%设置新字体
\newfontfamily{\fontcourier}{Courier New} 			% courier New
%\newfontfamily{\fontkai}{KaiTi}					% 楷体,我觉得最漂亮的中文字体
%\newfontfamily{\fontsong}{SimSun}					% 宋体,很难看但是很常用的中文字体
%\newfontfamily{\fonthei}{SimHei}					% 黑体
\newfontfamily{\fontarial}{Arial}					% Arial
%\newfontfamily{\fonthelvetica}{Helvetica}			% Helvetica
%\newfontfamily{\fontconsolas}{Consolas}				% Consolas
%\newfontfamily{\fonttimes}{Times}					% Times
\newfontfamily{\fontmenlo}{Menlo}					% Menlo
%\newfontfamily{\fontgillsnas}{Gill Sans}				% Gill Sans

% end of font setting

\raggedright

\pagenumbering{arabic}

\def\bull{\vrule height 0.8ex width .7ex depth -.1ex }
% DEFINITIONS FOR RESUME

\newenvironment{changemargin}[2]{%
  \begin{list}{}{%
    \setlength{\topsep}{0pt}%
    \setlength{\leftmargin}{#1}%
    \setlength{\rightmargin}{#2}%
    \setlength{\listparindent}{\parindent}%
    \setlength{\itemindent}{\parindent}%
    \setlength{\parsep}{\parskip}%
  }%
  \item[]}{\end{list}
}

\newcommand{\lineover}{
	\begin{changemargin}{-0.05in}{-0.05in}
		\vspace*{-8pt}
		\hrulefill \\
		\vspace*{-2pt}
	\end{changemargin}
}

\newcommand{\header}[1]{
	\begin{changemargin}{-0.5in}{-0.5in}
	\fontsize{12}{14} \scshape{\textbf{#1}}\\
  	%\lineover
	\end{changemargin}
}

\newcommand{\contact}[3]{
	\begin{changemargin}{-0.5in}{-0.5in}
		{\fontsize{15}{18} \scshape \textbf{#1}}\\ \smallskip
		\lineover
		\begin{flushright}
			{\fontarial \emph{#2}}\\ \smallskip
			{\fontarial \emph{#3}}\smallskip
		\end{flushright}
	\end{changemargin}
}

\newenvironment{body} {
	\vspace*{-16pt}
	\begin{changemargin}{-0.5in}{-0.5in}
  }	
	{\end{changemargin}
}	

\newcommand{\school}[4]{
	\textbf{#1} \hfill \emph{#2\\}
	#3\\ 
	#4\\
}

% END RESUME DEFINITIONS

\begin{document}

%%%%%%%%%%%%%%%%%%%%%%%%%%%%%%%%%%%%%%%%%%%%%%%%%%%%%%%%%%%%%%%%%%%%%%%%%%%%%%%%
% Name
\contact{王文斌}{ecnuwbwang@gmail.com}{(+86)151-2101-7457}

\renewcommand{\baselinestretch}{1.2} \normalsize

%%%%%%%%%%%%%%%%%%%%%%%%%%%%%%%%%%%%%%%%%%%%%%%%%%%%%%%%%%%%%%%%%%%%%%%%%%%%%%%%
% 教育经历
\header{教育经历}

\begin{body}
	\vspace{14pt}
	{华东师范大学 计算机应用技术 硕士} \hfill {\fontarial 2012/09 - 2015/06} \\
	%\smallskip
	{华东师范大学 计算机科学与技术 学士} \hfill {\fontarial 2008/09 - 2012/06} \\
\end{body}

\medskip

%%%%%%%%%%%%%%%%%%%%%%%%%%%%%%%%%%%%%%%%%%%%%%%%%%%%%%%%%%%%%%%%%%%%%%%%%%%%%%%%
% 工作经历
\header{工作经历}
\begin{body}
	\vspace{14pt}
	\textbf{苏州美能华智能科技} \hfill 全栈工程师 {\fontarial 2019/04 - }\\
	\smallskip
	\begin{justify}
	作为团队组长和主力研发,我从零开始参与了多个产品和服务的研发,包括智能客服系统、小星助手(\url{https://xiaoxing.meinenghua.com})、星原AI+RPA低代码平台,以及上层的应用引擎、智能审批等应用层服务。我的主要职责和成就包括:
	\end{justify}
	\vspace*{-2pt}
	\begin{itemize} \itemsep -0pt  % reduce space between items
		\item 作为组长,负责需求沟通、系统设计、决策设计方案和制订研发计划,带领团队(约5人)完成产品开发。我能够在功能、技术和成本等众多因素之间做出权衡和取舍。\\
	\end{itemize}
	\vspace*{-10pt}
	\begin{itemize} \itemsep -0pt  % reduce space between items
		\item 作为研发人员,参与设计和开发了星原AI+RPA低代码平台,以及相关的应用层服务。\\
	\end{itemize}
	\vspace*{-10pt}
	\begin{itemize} \itemsep -0pt  % reduce space between items
		\item 根据小星助手的产品功能需求,负责系统设计和后端开发,并指导前端开发。\\
	\end{itemize}
%	\vspace*{-10pt}
%	\begin{itemize} \itemsep -0pt  % reduce space between items
%		\item 针对B端客户的集成现有用户认证服务的需求,我设计和实现了第三方登录方案。\\
%	\end{itemize}
	\vspace*{-10pt}
	\begin{itemize} \itemsep -0pt  % reduce space between items
		\item 针对B端客户大量的私有化部署需求,我开发了一套部署工具,实现了离线部署。利用nginx等服务,提高了部署docker镜像的适用性,降低了前端代码配置文件的复杂度。\\
	\end{itemize}	
	\vspace*{-4pt}
	涉及技术:{\fontarial Vue.js, Node.js, PostgreSQL, Redis, Nginx, Docker, Kubernetes}\\

	\vspace{10pt}
	\textbf{微软(苏州)} \hfill 软件工程师 {\fontarial 2015/07 - 2019/03}\\ 
	\smallskip
	本人在{\fontarial Bing Ads}部门的{\fontarial UCM}团队工作。{\fontarial UCM}是一款供广告销售团队使用的用户管理工具。本人的主要工作内容:\\
	%\smallskip
	\vspace*{-2pt}
	\begin{itemize} \itemsep -0pt  % reduce space between items
		\item 前端的功能开发、维护和性能优化,参与优化{\fontarial SPA First Load Time}过长,局部刷新性能优化。\\
	\end{itemize}
	\vspace*{-10pt}
	\begin{itemize} \itemsep -0pt  % reduce space between items
		\item 基于{\fontarial UCM}前端的代码框架,增强了前端单元测试框架,降低开发成本。\\
	\end{itemize}
	\vspace*{-10pt}
	\begin{itemize} \itemsep -0pt  % reduce space between items
		\item 参与前端技术更新,将{\fontarial React}和{\fontarial UCM}的现有代码整合,使它们能够一起工作。\\
	\end{itemize}
	\vspace*{-4pt}
	涉及技术:{\fontarial jQuery.js, Bootstrap.js, React}\\
	% \textbf{智能客服多轮对话系统} \hfill {\fontarial 2018/06 - 2018/09}\\ 
	% 该系统是基于意图识别和填槽技术的面向任务的对话系统,采用微服务架构,包含对话任务设计、对话模板管理和对话逻辑处理三部分。该系统提供{\fontarial Web UI}允许管理员自定义对话任务模板,并提供聊天窗口让用户与机器人进行多轮对话。本人负责开发对话模板管理的{\fontarial Web UI}部分,使用{\fontarial React}开发。\\

	\vspace{10pt}
	\textbf{百度} \hfill 实习生 {\fontarial 2013/07 - 2013/10}\\ 
	\smallskip
	%\textbf{广告管家} \hfill {\fontarial 2013/07 - 2013/10}\\ 
	%\smallskip
	参与百度广告管家{\fontarial 2.0}版本业务端的部分测试工作,使用{\fontarial Selenium}完成自动化测试用例的开发。
	%\smallskip

	\vspace{10pt}
	\textbf{谷歌-企业社会责任部} \hfill 实习生 {\fontarial 2012/02 - 2012/08}\\ 
	\smallskip
	%\textbf{益暖中华{\fontarial (G1C1)}} \hfill {\fontarial 2012/02 - 2012/08}\\ 
	%\smallskip
	参与益暖中华{\fontarial (G1C1)}项目,主要负责{\fontarial Java+MySQL}以及{\fontarial PHP+MySQL}的网站开发工作。\\ 
	%\smallskip
	\vspace*{-6pt}
	\begin{itemize} \itemsep -0pt  % reduce space between items
		\item 负责益暖中华{\fontarial(www.gong1chuang1.com)}网站日常维护和功能升级。\\
	\end{itemize}
	\vspace*{-12pt}
	\begin{itemize} \itemsep -0pt  % reduce space between items
		\item 参与完成益暖中华主网站的迁移,从{\fontarial Java Web}迁移到{\fontarial PHP}。\\
	\end{itemize}

\end{body}

\medskip

%%%%%%%%%%%%%%%%%%%%%%%%%%%%%%%%%%%%%%%%%%%%%%%%%%%%%%%%%%%%%%%%%%%%%%%%%%%%%%%%
% 项目经历
\header{项目经历}

\begin{body}
 	\vspace{14pt}
  	\textbf{智能客服系统} \\
  	\smallskip
	该系统可以让用户训练自己的智能机器人客服,结合给机器人配置的人工客服,提供整套的智能客服解决方案。本人带领团队(2人)从零开始设计和开发了除AI算法之外的整套系统,本人主要工作包括:
	\vspace*{-2pt}
	\begin{itemize} \itemsep -0pt  % reduce space between items
		\item 用户认证和权限模块,支持集成客户的用户体系登录;\\
	\end{itemize}
	\vspace*{-10pt}
	\begin{itemize} \itemsep -0pt  % reduce space between items
		\item 机器人管理端,拥有机器人和人工客服管理功能,问答库和词库管理功能,对话历史消息的标注和训练功能,数据报表功能;\\
	\end{itemize}
	\vspace*{-10pt}
	\begin{itemize} \itemsep -0pt  % reduce space between items
		\item 用户端和客服端,利用WebSocket实现双向通信,利用Redis维护用户和客服状态,会话状态,实现人工客服分配,用户空闲检测等功能;\\
	\end{itemize}
	\vspace*{-10pt}
	\begin{itemize} \itemsep -0pt  % reduce space between items
		\item 机器人对话和对话引擎中间层,用于统一底层算法不同引起的不一致性,方便接入和替换底层算法;\\
	\end{itemize}
	\vspace*{-10pt}
	\begin{itemize} \itemsep -0pt  % reduce space between items
		\item 实现了离线数据报表服务,在牺牲一定时效性(5分钟)的情况下,提升了数据报表的查询性能。\\
	\end{itemize}

	\vspace{14pt}
  	\textbf{小星助手服务} \\
  	\smallskip
 	该服务基于大语言模型,可以让用户通过上传文档,让机器人学习文档内容,进而提供基于文档内容的问答服务。本人主要工作包括:
	\vspace*{-2pt}
	\begin{itemize} \itemsep -0pt  % reduce space between items
		\item 围绕市场开拓的需求,设计产品功能,制订开发计划,确保产品的快速迭代;\\
	\end{itemize}
	\vspace*{-10pt}
	\begin{itemize} \itemsep -0pt  % reduce space between items
		\item 负责数据库设计和后端开发;\\
	\end{itemize}
	\vspace*{-10pt}
	\begin{itemize} \itemsep -0pt  % reduce space between items
		\item 带领团队重构前端代码,把手机端网页和PC端网页代码合并为一套;\\
	\end{itemize}
	\vspace*{-10pt}
	\begin{itemize} \itemsep -0pt  % reduce space between items
		\item 保证开发进度的同时,完善之前为了赶时间舍弃的部分。\\
	\end{itemize}
	
	\vspace{14pt}
  	\textbf{星原AI+RPA低代码平台} \\
  	\smallskip
 	该平台提供了图形化的BPMN流程图制作工具(星原Studio),丰富的AI模块和运行环境,RPA指令和运行环境,让普通的实施工程师也能快速制作满足客户需求的AI+RPA流程,并发布成AI应用或者RPA任务。降低了AI算法和RPA流程的开发成本。该平台由10多个部分组成,本人负责管理中心、用户中心、星原商店、应用引擎、数据报表、发布工具等6个服务的研发工作,其余部分有AI运行时、RPA管理中心、RPA工作机、星原Studio、星原助手等。本人还参与了AI运行时v3版设计工作,此版为了提高GPU的利用率和AI运行时的并发量,将GPU的执行粒度从流程图级别降低到模块级别,但由于公司发展方向问题,最终并没有实现。\\
	
	\vspace{14pt}
  	\textbf{智能审核服务} \\
  	\smallskip
 	该服务主要面向政府的审批机构,它能利用AI算法把非结构化数据(比如:图片文件)转换成结构化数据,再配合审核规则,能够迅速判断一组文件是否符合所有规则要求,从而降低人工成本。本人主要工作包括:
	\vspace*{-2pt}
	\begin{itemize} \itemsep -0pt  % reduce space between items
		\item 系统设计,对外API设计与开发;\\
	\end{itemize}
	\vspace*{-10pt}
	\begin{itemize} \itemsep -0pt  % reduce space between items
		\item 通用UI设计与开发;\\
	\end{itemize}
	\vspace*{-10pt}
	\begin{itemize} \itemsep -0pt  % reduce space between items
		\item 按照客户需求进行UI定制化开发。\\
	\end{itemize}

\end{body}

\medskip

%%%%%%%%%%%%%%%%%%%%%%%%%%%%%%%%%%%%%%%%%%%%%%%%%%%%%%%%%%%%%%%%%%%%%%%%%%%%%%%%
% 获奖经历
\header{获奖经历}
\begin{body}
	\vspace{14pt}
	{{\fontarial ACM/ICPC} 国际大学生程序设计竞赛福州赛区 铜奖} \hfill {\fontarial 2011/11}\\
	%\smallskip
	{{\fontarial ACM/ICPC} 国际大学生程序设计竞赛北京赛区 铜奖} \hfill {\fontarial 2011/10}\\
	%\smallskip
	{{\fontarial MCM} 美国大学生数学建模大赛 二等奖} \hfill {\fontarial 2011/02}\\
	%\smallskip
%	{{\fontarial ACM/ICPC} 国际大学生程序设计竞赛天津赛区 铜奖} \hfill {\fontarial 2010/10}\\
\end{body}

\medskip


%%%%%%%%%%%%%%%%%%%%%%%%%%%%%%%%%%%%%%%%%%%%%%%%%%%%%%%%%%%%%%%%%%%%%%%%%%%%%%%%
% 项目经历
%\header{项目经历}

%\begin{body}
%	\vspace{14pt}
	
	%{\textbf{自动查询扩展技术研究 研究课题}}{} \hfill \emph{\fontarial 2014/03 - }至今\\
	%\smallskip
	%项目描述:本课题目标是通过查询扩展技术,提高信息检索的准确率。\\
	%\medskip
	%\vspace*{-6pt}
	%\begin{itemize} \itemsep -0pt  % reduce space between items
	%	\item 参与研究查询词权重调整\emph{\fontarial(query reweighting)}技术,提出了一种基于全局分析的权重计算公式。\\
	%\end{itemize}
	%\vspace*{-12pt}
	%\begin{itemize} \itemsep -0pt  % reduce space between items
	%	\item 技术环境:\emph{\fontarial Mac Indri}\\
	%\end{itemize}
	
%	{\textbf{基于本体的语意搜索引擎的开发与研究 研究课题}}{} \hfill {\fontarial 2012/12 - 2013/06}\\
	%\smallskip
%	项目描述:本课题目标是将语义分析的技术引入到搜索引擎,改善当前搜索引擎的搜索效果,提出基于本体的语意搜索引擎模型,来提高搜索结果的准确率。\\
	%\medskip
%	\vspace*{-6pt}
%	\begin{itemize} \itemsep -0pt  % reduce space between items
%		\item 参与搜索引擎模块的研究,除了使用现成的{\fontarial Google}和{\fontarial Bing},采用{\fontarial Hadoop}云环境和{\fontarial Nutch}搜索引擎。\\
%	\end{itemize}
%	\vspace*{-12pt}
%	\begin{itemize} \itemsep -0pt  % reduce space between items
%		\item 技术环境:{\fontarial Linux Nutch Hadoop}\\
%	\end{itemize}
	
%	{\textbf{{\fontarial Small C}编译器 课程设计}}{} \hfill {\fontarial 2011/10 - 2011/12}\\
%	项目描述:根据文法,使用{\fontarial LL(1)}方法,实现一个{\fontarial Small C}编译器。\\
%	\vspace*{-6pt}
%	\begin{itemize} \itemsep -0pt  % reduce space between items
%		\item 在课程要求的基础上,新增了一些语法的实现,有比较完善的编译报错机制。\\
%	\end{itemize}
%	\vspace*{-12pt}
%	\begin{itemize} \itemsep -0pt  % reduce space between items
%		\item 技术环境:{\fontarial Java}\\
%	\end{itemize}
	
%	{\textbf{数独游戏 课程设计}}{} \hfill \emph{\fontmenlo2010/10 - 2010/12}\\
%	项目描述:在传统的 {\fontmenlo 9×9} 数独游戏基础上,定制规则。该小软件可以自动生成游戏局面,提供答案,还支持手动输入数独,帮助用户解答。\\
%	\vspace*{-6pt}
%	\begin{itemize} \itemsep -0pt  % reduce space between items
%		\item 负责游戏界面的设计开发,后台解答模块算法实现。\\
%	\end{itemize}
%	\vspace*{-12pt}
%	\begin{itemize} \itemsep -0pt  % reduce space between items
%		\item 技术环境:\emph{\fontmenlo C\raise.3ex\hbox{\small{\#}} .NET}\\
%	\end{itemize}
%\end{body}

%\medskip

% \newpage{} % uncomment this line if you want to force a new page

%%%%%%%%%%%%%%%%%%%%%%%%%%%%%%%%%%%%%%%%%%%%%%%%%%%%%%%%%%%%%%%%%%%%%%%%%%%%%%%%
% 自我评价
\header{自我评价}

\begin{body}
	\vspace{14pt}
	具有8年以上的软件开发经验,5年以上的全栈开发经验,拥有丰富的系统设计经验,能够在功能、技术和研发成本等众多因素之间做出权衡和取舍。\\
	具有4年以上的领导团队经验,能够带领一个团队出色的完成研发计划。\\
	做事专注,认真负责,喜欢钻研,相信技术改变世界。\\
	开发技能:{\fontarial JavaScript}(熟练) {\fontarial Python}(熟悉) Go(学习中)\\
	语言技能:普通话(母语) 英语({\fontarial CET-6})\\
\end{body}

\medskip

%%%%%%%%%%%%%%%%%%%%%%%%%%%%%%%%%%%%%%%%%%%%%%%%%%%%%%%%%%%%%%%%%%%%%%%%%%%%%%%%
% 自我评价
%\header{自我评价}

%\begin{body}
%	\vspace{14pt}
%	做事专注,喜欢钻研,相信技术改变世界。\\
%	乐观开朗,热爱生活,一个发散正能量的人。\\
%\end{body}

%\medskip


%%%%%%%%%%%%%%%%%%%%%%%%%%%%%%%%%%%%%%%%%%%%%%%%%%%%%%%%%%%%%%%%%%%%%%%%%%%%%%%%
% 兴趣爱好
\header{兴趣爱好}

\begin{body}
	\vspace{14pt}
	烹饪 羽毛球 看电影\\
\end{body}

\end{document}
